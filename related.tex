\section{Related Work}
In past decades information network analysis has become a hot research topic in data mining and information retrieval fields \cite{shi2017survey}. There are various research areas that focus on learning heterogeneous information networks that include clustering \cite{huang2016meta}, classification \cite{kong2012meta}, link prediction \cite{zhang2017link}, similarity \cite{sun2011pathsim}, etc. Although, these algorithmic methods are well defined on complex heterogeneous meta-structures, it require algorithmic processing of complete graph. Alternatively, information retrieval algorithms from HINs proposes random walk models to evaluate relevance of different-typed objects that requires defined data structure and node types. Furthermore, these traditional approaches does not work with expanded node links because of generic node labels without creating direct links.

In recent years automatic feature learning algorithms are at the forefront of machine learning research and deep learning has been successfully adopted in many applications such as speech recognition and image classification. Furthermore, graph clustering methods based on deep neural network \cite{tian2014learning} have produced promising results. While CNNs are very successful in various supervised learning applications, it falls behind in AI related applications. Deep reinforcement learning (RL) models are major improvement over traditional deep learning in terms of capacity, training time and sample size \cite{zambaldi2018relational}. Reinforcement learning is efficiently utilized in graph structures \cite{liang2017deep}. CapsNet resolves issues with CNNs by applying dynamic routing between capsules. Our novel graph embedding model is based on CapsNet for unsupervised learning on knowledge graphs generated from heterogeneous data sources because of localized characteristics of small independent graph structures.