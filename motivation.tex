\section{Motivation}
Data sets generated from social networks, crowd sourcing, IoT devices and predictive analysis consists of large number of interconnected, multi-typed components that is modelled as heterogeneous information networks (HINs). Traditional methods of analyzing HINs over such huge amount of data sets makes it impractical to use. Furthermore, social networks and crowd sourced databases could polluted with machine generated data that makes traditional graph traversal methods for data mining and analysis unusable unless additional work is performed to sanitize input data. Additionally, traditional methods to generate Knowledge Base from HINs requires computations in the order of magnitude of nodes and edges. Similarly, analysis images and text generated from these platforms is also a challenge that has been successfully overcome by using neural network based machine learning methods on massively distributed systems with high speed memory and processors. Capsule networks is one of deep learning method that has produced promising results by overcoming short falls of CNN. These advancements in deep learning and success of CapsNet is our motivation behind creating algorithm for applying CapsNet architecture for knowledge base of diverse nature with geo-spatial information.